\subsection{Input Circuit}
In Europe the mains voltage in houses is 230 volts, all of the input circuit before the transformer needs to be designed to that voltages. 
As this part of the input circuit we want to besure that the noise on mains volts do not propegate to our output volts, 
we are going to use a main emi filter to filter out some of the noise on the mains system. 
This will also double as filter for any noise we are going to send out to the mains voltage. 

As a means for input protection we are also going to use fuse (see formula \ref{eq:transfer-transformer} on page \pageref{eq:transfer-transformer}).

\subsubsection{Transformer}
The transfer formula for the transformer 
\begin{align}\label{eq:transfer-transformer}
m &= \frac{I_s}{I_p} = \frac{V_p}{V_s}
\end{align}
For the input supply I am going to use a transformer 2x18V at 1.5A, that I already have. And I am going to connect the 18V's in series to get 36V at 1.5A. But for finding the transfer value we can use formula \ref{eq:transfer-transformer}.
\begin{align}   
m = \frac{V_p}{V_s} = \frac{230V}{18V} = 12.78 \nonumber
\end{align}
 To find out which size of fuse that we are going to be using, we need to find out how much current that are going to be used on the main side of the transformer.
We are going rearrange in formula \ref{eq:transfer-transformer} as we know what $I_s$ is and we have calculated $m$.
\begin{align}
m = \frac{I_s}{I_p} \nonumber \\
\Rightarrow \nonumber \\
I_p \cdot m = I_p \cdot \frac{I_s}{I_p} = I_s \nonumber \\
\Rightarrow \nonumber \\
\frac{ I_p \cdot m}{m} = \frac{I_s}{m} \nonumber \\
\Rightarrow \nonumber \\
I_p = \frac{I_s}{m} \label{eq:found-Ip}
\end{align}
Now we can calculate $I_p$ with $I_s$ and $m$ from \ref{eq:found-Ip}.
\begin{align}
I_p = \frac{I_s}{m} = \frac{1.5A}{12.78} = 0.117A = \underline{\underline{117mA}} \nonumber 
\end{align}
So the primary side of the transformer needs a fuse of 117mA, 
but as I can not find that size, the one that is going to be used is a 125mA.
On the secondary side, I am going to use a 1,5A fuse.

\subsubsection{DC voltage}
Now we should calculate if we want to use 2x18V or put them in serial and get 36V. 
One thing to keep in mind, is that when using a bridge rectifier the input frequency is doubled.
\paragraph{36 \vac}
The 36 volt ac are going to be converted to dc, via fullwave diode bridge. 
Sinces the ac voltage specified on the transformer is RMS voltage, and not the peak voltage. 
We can calculate what the dc voltage will become assuming no losses.
\begin{align} \label{eq:calc_vdc}
\text{V}_{ \text{DC} } &= \frac{ 2 \cdot \text{V}_{ \text{max} } }{ \pi } \\
         &= 0.637 \cdot \text{V}_{ \text{max} } \\
         &= 0.9 \cdot \text{V}_{ \text{RMS} } \\
         &= 0.9 \cdot 36 \text{V}_{\text{AC}}\nonumber \\
         &= 32.4 \text{V}_{ \text{DC} } \nonumber
\end{align}
But since we are going to use a bridge rectifier there is two times diode voltage drop that gives total drop of \(1.4V\). 
Which meaning that we are properly only going to see around \( 31V \) if measure at the diode bridge output. 
\paragraph{18 \vac}
\begin{align} \label{eq:calc_18vdc}
    \text{V}_{ \text{DC} }  &= \frac{ 2 \cdot \text{V}_{ \text{max} } }{ \pi } \\
             &= 0.637 \cdot \text{V}_{ \text{max} } \\
             &= 0.9 \cdot \text{V}_{ \text{RMS} } \\
             &= 0.9 \cdot 18 \text{V}_{ \text{AC} } \nonumber \\
             &= 16.2 \text{V}_{ \text{DC} } \nonumber
\end{align}
And again we need to take into account the diode drop, which gives 14.8V after the diode bridge.

\subsection{Ripple voltage}
The diode bridge does not create a dc voltage, but rather flips over the negativ half of the sinus wave to the positive. So there is a need for smoothering out the sinus wave to get a dc voltage. 
We need to figure out how much ripple voltage, that we can get away with at this point. We are going to use formula \ref{eq:calc_vripple} to calculate our $V_{ripple}$.
\begin{align}\label{eq:calc_vripple}
\text{V}_{ \text{PP}_{\text{ripple} }} &= \frac{ \text{I}_{ \text{load} } }{ 2 \cdot f \cdot C } [V]
\end{align}
If we choose a big capacitor, then the time for it going to be discharge from fully charged is going to be long. As long there is voltage on the capacitor the rest of the circuit is working. At table \ref{tab:choose-c} I have made some calculation of a couple of capacitor that I can buy.
\begin{table}[ht]
\centering
\caption{Choosing smooting capacitor} \label{tab:choose-c}
\begin{tabular}{|c|r|r|r|r|} 
\hline
C & $V_{ripple}$ \@ 1.5A & $V_{ripple}$ \@ 1A & $V_{ripple}$ \@ 500mA & $V_{ripple}$ \@ 10mA \\ \hline \hline
 47 uF  &  3.525 mV &  2.350 mV &  1.175 mV & 0.024 mV \\ \hline
100 uF  &  7.500 mV &  5.000 mV &  2.500 mV & 0.050 mV \\ \hline
120 uF  &  9.000 mV &  6.000 mV &  3.000 mV & 0.060 mV \\ \hline
150 uF  & 11.250 mV	&  7.500 mV &  3.750 mV & 0.075 mV \\ \hline
180 uF  & 13.500 mV &  9.000 mV &  4.500 mV & 0.090 mV \\ \hline
220 uF  & 16.500 mV & 11.000 mV &  5.500 mV & 0.110 mV \\ \hline
270 uF  & 20.250 mV & 13.500 mV &  6.750 mV & 0.135 mV \\ \hline
330 uF  & 24.750 mV & 16.500 mV &  8.250 mV & 0.165 mV \\ \hline
390 uF  & 29.250 mV & 19.500 mV &  9.750 mV & 0.195 mV \\ \hline
470 uF  & 35.250 mV & 23.500 mV & 11.750 mV	& 0.235 mV \\ \hline
560 uF  & 42.000 mV & 28.000 mV	& 14.000 mV & 0.280 mV \\ \hline
680 uF  & 51.000 mV & 34.000 mV & 17.000 mV & 0.340 mV \\ \hline
820 uF  & 61.500 mV & 41.000 mV & 20.500 mV & 0.410 mV \\ \hline
\end{tabular}
\end{table}